\documentclass[polish,polish,a4paper]{article}
\usepackage[T1]{fontenc}
\usepackage[utf8]{inputenc}
\usepackage{pslatex}
\usepackage{setspace}
\usepackage{caption}
\usepackage{amssymb}
\usepackage{amsmath}
\usepackage{anysize}
\usepackage{graphicx}
\usepackage{hyperref}
\usepackage{float}
\usepackage[polish]{babel}
\hypersetup{
	colorlinks=true,
	linkcolor=blue,
	filecolor=magenta,      
	urlcolor=cyan,
}
\marginsize{2.5cm}{2.5cm}{2cm}{2cm}


\begin{document}
	
\begin{spacing}{1.5}
		\begin{titlepage}
		\vspace*{\fill}
		\begin{center}
			{\Large Algorytmy i struktury danych \\[0.1cm]
				Sprawozdanie z zadania w zespołach nr. 1\\[0.1cm]
				prowadząca: dr hab. inż. Małgorzata Sterna, prof PP}\\[0.7cm]
			{\huge Algorytmy sortujące\\ [0.7cm]}
			{\large autorzy:\\[0.1cm]}
			{\Large Piotr Więtczak nr indeksu 132339,\\[0.1cm] Tomasz Chudziak nr indeksu 136691}\\[0.5cm]
			\today
		\end{center}
		\vspace*{\fill}
	\end{titlepage}
	
	\section{Implementacja algorytmów sortujących}
	Do implementacji metod sortowania posłużyliśmy się językiem $ C++ $, każda metoda została napisana  w odrębnej  funkcji, która za parametry przyjmuje kolejno: wskaźnik na tablicę, rozmiar sortowanej tablicy oraz jako ostatni wartość opcjonalną “reverse” typu bool, która odpowiada za to czy tablica będzie posortowana malejąco. Do mierzenia czasu poszczególnych metod użyliśmy klasy $ std::chrono::high::resolution\_clock  $ z biblioteki $ chrono $.
	\section{Badana zależność czasu obliczeń $ t[s]$ od liczby sortowanych elementów~$ n $. }
	W celu lepszego przedstawienia otrzymanych danych podzieliliśmy metody na dwie grupy, "wolne" (Insertion Sort, Selection Sort, Bubble Sort) i "szybkie" (Counting Sort, Quick Sort, Merge Sort, Heap Sort).
	
	\subsection{Metody $ "$wolne$" $}
	\subsubsection{Opis algorytmów $"$wolnych$"$}
	\subsubsection*{Insert Sort}
	Zalety:
	\begin{itemize}
		\item działa w miejscu
		\item stabilny 
	\end{itemize}
	Wady:
	\begin{itemize}
		\item 
	\end{itemize}
	Inne cechy:
	\begin{itemize}
		\item zachowanie naturalne
	\end{itemize}
	
				\subsubsection*{Tabela przedstawiająca złożoność obliczeniową dla przypadków optymistycznego, średniego i pesymistycznego} 
	
	\begin{figure}[H]
		\begin{equation*}
		\begin{array}{|l|c|c|c|}
		\hline
		&$złożoność obliczeniowa$&$złożoność obliczeniowa$&$złożoność obliczeniowa$\\
		&$dla przypadku$&$dla przypadku$&$dla przypadku$\\
		&$optymistycznego$&$średniego$&$pesymistycznego$\\
		\hline
		$Insert Sort$&O(n^2)&O(n^2)&O(n^2)\\
		\hline
		\end{array}
		\end{equation*}
		\captionof{table}{Tablica złożoności obliczeniowej dla metody Insert Sort}
	\end{figure}
		
		\subsubsection*{Selection Sort}
	Zalety:
	\begin{itemize}
		\item działa w miejscu
		\item stabilny 
	\end{itemize}
	Wady:
	\begin{itemize}
		\item 
	\end{itemize}
	Inne cechy:
	\begin{itemize}
		\item zachowanie naturalne
	\end{itemize}
	
				\subsubsection*{Tabela przedstawiająca złożoność obliczeniową dla przypadków optymistycznego, średniego i pesymistycznego} 
	\begin{figure}[H]

		\begin{equation*}
		\begin{array}{|l|c|c|c|}
		\hline
		&$złożoność obliczeniowa$&$złożoność obliczeniowa$&$złożoność obliczeniowa$\\
		&$dla przypadku$&$dla przypadku$&$dla przypadku$\\
		&$optymistycznego$&$średniego$&$pesymistycznego$\\
		\hline
		$Insert Sort$&O(n^2)&O(n^2)&O(n^2)\\
		\hline
		\end{array}
		\end{equation*}
		\captionof{table}{Tablica złożoności obliczeniowej dla metody Selection Sort}
	\end{figure}
	
	
			\subsubsection*{Bubble Sort}
	Zalety:
	\begin{itemize}
		\item działa w miejscu
		\item stabilny 
	\end{itemize}
	Wady:
	\begin{itemize}
		\item 
	\end{itemize}
	Inne cechy:
	\begin{itemize}
		\item zachowanie naturalne
	\end{itemize}
	
	
	\subsubsection*{Tabela przedstawiająca złożoność obliczeniową dla przypadków optymistycznego, średniego i pesymistycznego} 
	
	\begin{figure}[H]
			\begin{equation*}
		\begin{array}{|l|c|c|c|}
		\hline
		&$złożoność obliczeniowa$&$złożoność obliczeniowa$&$złożoność obliczeniowa$\\
		&$dla przypadku$&$dla przypadku$&$dla przypadku$\\
		&$optymistycznego$&$średniego$&$pesymistycznego$\\
		\hline
		$Insert Sort$&O(n^2)&O(n^2)&O(n^2)\\
		\hline
		\end{array}
		\end{equation*}
		\captionof{table}{Tablica złożoności obliczeniowej dla metody Bubble Sort}
	\end{figure}

	\subsubsection*{Tabela ilustrująca zależności czasu sortowania od ilości elementów dla metod $"$wolnych$"$, zakres liczb $ [1,n] $.}
	
	\begin{figure}[H]
			\begin{equation*}
		\begin{array}{|c|c|c|c|}
		\hline
		&$Insertion Sort$&$Selection Sort$&$Bubble Sort$\\
		\hline
		\end{array}
		\end{equation*}
		\captionof{table}{Wyniki badań zależności czasu od iloci elementów dla metod $"$wolnych$"$}
	\end{figure}
	
	\subsubsection*{Wykres ilustrujący zależności czasu sortowania od ilości elementów dla metod $"$wolnych$"$, zakres liczb $ [1,n] $.}
	
	\subsection{Metody $"$szybkie$"$}
\end{spacing}
	
	\newpage
	\tableofcontents
\end{document}


