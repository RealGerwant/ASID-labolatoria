\documentclass[polish,polish,a4paper]{article}
\usepackage[T1]{fontenc}
\usepackage[utf8]{inputenc}
\usepackage{pgfplots}
\usepackage{pslatex}
\usepackage{setspace}
\usepackage{caption}
\usepackage{amssymb}
\usepackage{amsmath}
\usepackage{anysize}
\usepackage{graphicx}
\usepackage{hyperref}
\usepackage{float}
\usepackage[polish]{babel}
\hypersetup{
	colorlinks=true,
	linkcolor=blue,
	filecolor=blue,      
	urlcolor=blue,
}

\marginsize{2.5cm}{2.5cm}{2cm}{2cm}


\begin{document}
	
\begin{spacing}{1.5}
		\begin{titlepage}
			\begin{flushright}
				{ Wtorki 16:50\\
					Grupa I3\\
					Kierunek Informatyka\\
					Wydział Informatyki\\
					Politechnika Poznańska}
			\end{flushright}
		\vspace*{\fill}
		\begin{center}
			{\Large Algorytmy i struktury danych \\[0.1cm]
				Sprawozdanie z zadania w zespołach nr. 2\\[0.1cm]
				prowadząca: dr hab. inż. Małgorzata Sterna, prof PP}\\
			{\Huge Wybrane złożone struktury danych\\ [0.4cm]}
			{\large autorzy:\\[0.1cm]}
			{\large Piotr Więtczak nr indeksu 132339\\[0.1cm] Tomasz Chudziak nr indeksu 136691}\\[0.5cm]
			\today
		\end{center}
		\vspace*{\fill}
	\end{titlepage}

\section{Opis implementacji i czasy tworzenia wybranych struktur danych}

\begin{figure}[H]
	\subsubsection*{Wykres przedstawiający średni czas tworzenia struktury, w milisekundach, od rozmiaru zestawu danych}
	\centering
	\begin{tikzpicture}
	\begin{axis}[
	width=0.9\textwidth,
	height = 0.9\textwidth,
	title={Czasy tworzenia wybranych struktur danych},
	xlabel={Liczba elementów},
	ylabel={Czas tworzenia w misekundach},
	scaled x ticks = false,
	xtick distance = 10000,
	x tick label style={/pgf/number format/fixed},
	xticklabel style = {rotate= 90},
	x label style={at={(axis description cs:0.5,-0.1)},anchor=north},
	%%
	scaled y ticks = false,
	y tick label style={/pgf/number format/fixed},
	ytick distance = 20000,
	y label style={at={(axis description cs:-0.05,0.7)},anchor=east},
	%%
	legend pos=north west,
	ymajorgrids=true,
	grid style=dashed,
	]
	%%
	\addplot[
	color=black,
	mark=square,
	]
	coordinates {
(10000,1.351)(20000,2.759)(30000,4.243)(40000,5.816)(50000,7.340)(60000,9.139)(70000,10.470)(80000,12.037)(90000,13.760)(100000,15.312)


	};
	%%
		\addplot[
	color=blue,
	mark=triangle,
	]
	coordinates {
(10000,1.842)(20000,3.481)(30000,5.208)(40000,7.114)(50000,8.808)(60000,10.958)(70000,12.170)(80000,14.395)(90000,15.595)(100000,17.661)





	};
		\addplot[
color=red,
mark=star,
]
coordinates {
(10000,7.560)(20000,15.225)(30000,22.645)(40000,36.227)(50000,44.441)(60000,55.291)(70000,65.380)(80000,73.278)(90000,84.278)(100000,95.074)


};
		\addplot[
color=green,
mark=star,
]
coordinates {
(10000,6.480)(20000,9.952)(30000,15.420)(40000,20.329)(50000,25.882)(60000,30.847)(70000,36.604)(80000,43.219)(90000,48.005)(100000,53.679)




};
	\legend{
		$C_{B}$,
		$C_{L}$,
		$C_{TR}$,
		$C_{TB}$,
	}
	%%
	\end{axis}
	\end{tikzpicture}
\end{figure}

\begin{figure}[H]
		\subsubsection*{Tabela przedstawiająca średni czas tworzenia struktury w milisekundach}
		\centering
		\begin{equation*}
		\begin{array}{|p|p|p|p|p|}
		\hline

		\end{array}
		\end{equation*}
\end{figure}
	
\end{spacing}
	\newpage
	\tableofcontents
\end{document}


