\documentclass[polish,polish,a4paper]{article}
\usepackage[T1]{fontenc}
\usepackage[utf8]{inputenc}
\usepackage{pgfplots}
\usepackage{pslatex}
\usepackage{setspace}
\usepackage{caption}
\usepackage{amssymb}
\usepackage{amsmath}
\usepackage{anysize}
\usepackage{graphicx}
\usepackage{hyperref}
\usepackage{float}
\usepackage[polish]{babel}
\hypersetup{
	colorlinks=true,
	linkcolor=blue,
	filecolor=blue,      
	urlcolor=blue,
}

\marginsize{2.5cm}{2.5cm}{2cm}{2cm}


\begin{document}
	
\begin{spacing}{1.5}
		\begin{titlepage}
			\begin{flushright}
				{ Wtorki 16:50\\
					Grupa I3\\
					Kierunek Informatyka\\
					Wydział Informatyki\\
					Politechnika Poznańska}
			\end{flushright}
		\vspace*{\fill}
		\begin{center}
			{\Large Algorytmy i struktury danych \\[0.1cm]
				Sprawozdanie z zadania w zespołach nr. 2\\[0.1cm]
				prowadząca: dr hab. inż. Małgorzata Sterna, prof PP}\\
			{\Huge Wybrane złożone struktury danych\\ [0.4cm]}
			{\large autorzy:\\[0.1cm]}
			{\large Piotr Więtczak nr indeksu 132339\\[0.1cm] Tomasz Chudziak nr indeksu 136691}\\[0.5cm]
			\today
		\end{center}
		\vspace*{\fill}
	\end{titlepage}

\section{Opis implementacji i czasy tworzenia wybranych struktur danych}
	\subsubsection*{Wykresy przedstawiające średni czas tworzenia struktury, w milisekundach, od rozmiaru zestawu danych}
%%WYKRES CZASU TWORZENIA
\begin{figure}[H]
	\centering
	\begin{tikzpicture}
	\begin{axis}[
	width=0.9\textwidth,
	height = 0.5\textwidth,
	title={Czasy tworzenia wybranych struktur danych},
	xlabel={Liczba elementów},
	ylabel={Czas tworzenia w misekundach},
	%
	scaled x ticks = false,
	xtick distance = 10000,
	x tick label style={/pgf/number format/fixed},
	xticklabel style = {rotate= 90},
		x label style={at={(axis description cs:0.5,-0.15)},anchor=north},
	%%
	ytick distance = 25,
	scaled y ticks = false,
	y tick label style={/pgf/number format/fixed},
	y label style={at={(axis description cs:-0.05,0.85)},anchor=east},
	%%
	legend pos=north west,
	ymajorgrids=true,
	grid style=dashed,
	]
	%%
\addplot[
	color=black,
	mark=*,
	]
	coordinates {
(10000,1.650)(20000,5.843)(30000,5.028)(40000,6.641)(50000,8.402)(60000,10.921)(70000,12.831)(80000,14.280)(90000,15.891)(100000,17.556)(110000,19.236)(120000,21.554)(130000,25.102)(140000,24.829)(150000,26.912)(160000,28.519)(170000,31.551)(180000,33.262)(190000,34.803)(200000,37.533)
	};
	%%
\addplot[
	color=blue,
	mark=triangle,
	]
	coordinates {
(10000,2.044)(20000,4.073)(30000,6.169)(40000,8.355)(50000,10.297)(60000,12.575)(70000,14.377)(80000,16.281)(90000,18.565)(100000,20.329)(110000,22.617)(120000,25.787)(130000,25.953)(140000,28.711)(150000,30.632)(160000,32.516)(170000,34.703)(180000,38.168)(190000,38.614)(200000,40.629)
	};
\addplot[
	color=red,
	mark=diamond,
	]
	coordinates {
(10000,9.945)(20000,22.923)(30000,32.196)(40000,46.658)(50000,57.789)(60000,76.041)(70000,94.813)(80000,102.981)(90000,125.854)(100000,130.962)(110000,154.307)(120000,170.075)(130000,184.363)(140000,202.444)(150000,215.974)(160000,245.860)(170000,280.478)(180000,264.447)(190000,285.930)(200000,303.575)
	};
\addplot[
	color=green,
	mark=square,
	]
	coordinates {
(10000,7.169)(20000,14.959)(30000,23.456)(40000,32.567)(50000,42.923)(60000,49.510)(70000,58.417)(80000,67.797)(90000,75.034)(100000,84.913)(110000,95.912)(120000,105.213)(130000,111.399)(140000,120.719)(150000,135.144)(160000,138.828)(170000,158.247)(180000,160.696)(190000,168.794)(200000,180.382)
	};
	\legend{
		$C_{B}$,
		$C_{L}$,
		$C_{TR}$,
		$C_{TB}$,
	}
	%%
	\end{axis}
	\end{tikzpicture}
\end{figure}

%%WYKRES CZASU TWORZENIA skala logarytmiczna
\begin{figure}[H]
	\centering
	\begin{tikzpicture}
	\begin{axis}[
	width=0.9\textwidth,
	height = 0.5\textwidth,
	title={Czasy tworzenia wybranych struktur danych, skala logarytmiczna},
	xlabel={Liczba elementów},
	ylabel={Czas tworzenia w misekundach},
	%
	scaled x ticks = false,
	xtick distance = 10000,
	x tick label style={/pgf/number format/fixed},
	xticklabel style = {rotate= 90},
	x label style={at={(axis description cs:0.5,-0.15)},anchor=north},
	%%
	ymode = log,
	scaled y ticks = false,
	y tick label style={/pgf/number format/fixed},
y label style={at={(axis description cs:-0.05,0.85)},anchor=east},
	%%
	legend pos=north west,
	ymajorgrids=true,
	grid style=dashed,
	]
	%%
	\addplot[
	color=black,
	mark=*,
	]
	coordinates {
		(10000,1.650)(20000,5.843)(30000,5.028)(40000,6.641)(50000,8.402)(60000,10.921)(70000,12.831)(80000,14.280)(90000,15.891)(100000,17.556)(110000,19.236)(120000,21.554)(130000,25.102)(140000,24.829)(150000,26.912)(160000,28.519)(170000,31.551)(180000,33.262)(190000,34.803)(200000,37.533)
	};
	%%
	\addplot[
	color=blue,
	mark=triangle,
	]
	coordinates {
		(10000,2.044)(20000,4.073)(30000,6.169)(40000,8.355)(50000,10.297)(60000,12.575)(70000,14.377)(80000,16.281)(90000,18.565)(100000,20.329)(110000,22.617)(120000,25.787)(130000,25.953)(140000,28.711)(150000,30.632)(160000,32.516)(170000,34.703)(180000,38.168)(190000,38.614)(200000,40.629)
	};
	\addplot[
	color=red,
	mark=diamond,
	]
	coordinates {
		(10000,9.945)(20000,22.923)(30000,32.196)(40000,46.658)(50000,57.789)(60000,76.041)(70000,94.813)(80000,102.981)(90000,125.854)(100000,130.962)(110000,154.307)(120000,170.075)(130000,184.363)(140000,202.444)(150000,215.974)(160000,245.860)(170000,280.478)(180000,264.447)(190000,285.930)(200000,303.575)
	};
	\addplot[
	color=green,
	mark=square,
	]
	coordinates {
		(10000,7.169)(20000,14.959)(30000,23.456)(40000,32.567)(50000,42.923)(60000,49.510)(70000,58.417)(80000,67.797)(90000,75.034)(100000,84.913)(110000,95.912)(120000,105.213)(130000,111.399)(140000,120.719)(150000,135.144)(160000,138.828)(170000,158.247)(180000,160.696)(190000,168.794)(200000,180.382)
	};
	\legend{
		$C_{B}$,
		$C_{L}$,
		$C_{TR}$,
		$C_{TB}$,
	}
	%%
	\end{axis}
	\end{tikzpicture}
\end{figure}



\begin{figure}[H]
		\subsubsection*{Tabela przedstawiająca średni czas tworzenia struktury w milisekundach}
		\centering
		\begin{equation*}
		\begin{array}{|p|p|p|p|p|}
		\hline

		\end{array}
		\end{equation*}
\end{figure}
	
	\section{Przeszukiwanie wybranych struktur danych}
	
\subsubsection*{Wykres przedstawiający średni czas przeszukiwania struktury, w milisekundach, od rozmiaru zestawu danych}
%czasy przeszukiwania
	\begin{figure}[H]

		\centering
		\begin{tikzpicture}
		\begin{axis}[
		width=0.9\textwidth,
		height = 0.5\textwidth,
		title={Czasy przeszukiwania wybranych struktur danych},
		xlabel={Liczba elementów},
		ylabel={Czas przeszukiwania w misekundach},
		%
		scaled x ticks = false,
		xtick distance = 10000,
		x tick label style={/pgf/number format/fixed},
		xticklabel style = {rotate= 90},
			x label style={at={(axis description cs:0.5,-0.15)},anchor=north},
	%%	ymode = log,
		%%
		ytick distance = 25000,%%%
		scaled y ticks = false,
		y tick label style={/pgf/number format/fixed},
	y label style={at={(axis description cs:-0.05,0.85)},anchor=east},
		%%
		legend pos=north west,
		ymajorgrids=true,
		grid style=dashed,
		]
		%%
		\addplot[
		color=black,
		mark=*,
		]
		coordinates {
(10000,116.347)(20000,434.923)(30000,963.589)(40000,1663.180)(50000,2755.070)(60000,3933.720)(70000,5211.100)(80000,6811.880)(90000,8601.650)(100000,10571.600)(110000,12800.800)(120000,15197.600)(130000,17988.800)(140000,20721.400)(150000,23809.200)(160000,27484.000)(170000,30563.800)(180000,34389.200)(190000,38219.600)(200000,42337.700)
		};
		%%
		\addplot[
		color=gray,
		mark=*,
		]
		coordinates {
(10000,4.534)(20000,11.445)(30000,15.227)(40000,20.992)(50000,26.554)(60000,33.620)(70000,41.562)(80000,46.460)(90000,52.667)(100000,58.985)(110000,66.020)(120000,72.712)(130000,82.524)(140000,87.497)(150000,96.570)(160000,102.584)(170000,109.273)(180000,114.565)(190000,123.794)(200000,132.252)
		};
		\addplot[
		color=blue,
		mark=triangle,
		]
		coordinates {
(10000,354.608)(20000,1435.100)(30000,3150.970)(40000,5964.640)(50000,17094.000)(60000,16052.900)(70000,21362.700)(80000,34108.400)(90000,45225.500)(100000,64175.800)(110000,79715.500)(120000,102809.000)(130000,131624.000)(140000,159801.000)(150000,183256.000)(160000,222896.000)(170000,244782.000)(180000,292448.000)(190000,318782.000)(200000,371754.000)
		};
\addplot[
		color=red,
		mark=diamond,
		]
		coordinates {
(10000,6.778)(20000,13.393)(30000,20.482)(40000,30.896)(50000,40.196)(60000,53.579)(70000,58.858)(80000,69.323)(90000,80.632)(100000,93.133)(110000,102.264)(120000,113.286)(130000,128.353)(140000,142.335)(150000,159.782)(160000,173.954)(170000,207.902)(180000,187.756)(190000,206.874)(200000,220.576)
		};
\addplot[
	color=green,
	mark=square,
	]
	coordinates {
(10000,5.125)(20000,12.102)(30000,16.104)(40000,26.580)(50000,31.822)(60000,38.924)(70000,48.745)(80000,53.877)(90000,67.380)(100000,71.823)(110000,79.161)(120000,91.825)(130000,103.391)(140000,115.199)(150000,127.070)(160000,126.894)(170000,157.062)(180000,147.043)(190000,157.062)(200000,168.860)
	};
		\legend{
			$S_{SB}$,
			$S_{BB}$,
			$S_{L}$,
			$S_{TR}$,
			$S_{TB}$,
		}
		%%
		\end{axis}
		\end{tikzpicture}
	\end{figure}
%czasy przeszukiwania skala logarytmiczna
	\begin{figure}[H]
	
	\centering
	\begin{tikzpicture}
	\begin{axis}[
	width=0.9\textwidth,
	height = 0.5\textwidth,
	title={Czasy przeszukiwania wybranych struktur danych, skala logarytmiczna},
	xlabel={Liczba elementów},
	ylabel={Czas przeszukiwania w misekundach},
	%
	scaled x ticks = false,
	xtick distance = 10000,
	x tick label style={/pgf/number format/fixed},
	xticklabel style = {rotate= 90},
		x label style={at={(axis description cs:0.5,-0.15)},anchor=north},
	ymode = log,
	%%
	scaled y ticks = false,
	y tick label style={/pgf/number format/fixed},
y label style={at={(axis description cs:-0.05,0.85)},anchor=east},
	%%
	legend pos=north west,
	ymajorgrids=true,
	grid style=dashed,
	]
	%%
	\addplot[
	color=black,
	mark=*,
	]
	coordinates {
		(10000,116.347)(20000,434.923)(30000,963.589)(40000,1663.180)(50000,2755.070)(60000,3933.720)(70000,5211.100)(80000,6811.880)(90000,8601.650)(100000,10571.600)(110000,12800.800)(120000,15197.600)(130000,17988.800)(140000,20721.400)(150000,23809.200)(160000,27484.000)(170000,30563.800)(180000,34389.200)(190000,38219.600)(200000,42337.700)
	};
	%%
	\addplot[
	color=gray,
	mark=*,
	]
	coordinates {
		(10000,4.534)(20000,11.445)(30000,15.227)(40000,20.992)(50000,26.554)(60000,33.620)(70000,41.562)(80000,46.460)(90000,52.667)(100000,58.985)(110000,66.020)(120000,72.712)(130000,82.524)(140000,87.497)(150000,96.570)(160000,102.584)(170000,109.273)(180000,114.565)(190000,123.794)(200000,132.252)
	};
	\addplot[
	color=blue,
	mark=triangle,
	]
	coordinates {
		(10000,354.608)(20000,1435.100)(30000,3150.970)(40000,5964.640)(50000,17094.000)(60000,16052.900)(70000,21362.700)(80000,34108.400)(90000,45225.500)(100000,64175.800)(110000,79715.500)(120000,102809.000)(130000,131624.000)(140000,159801.000)(150000,183256.000)(160000,222896.000)(170000,244782.000)(180000,292448.000)(190000,318782.000)(200000,371754.000)
	};
	\addplot[
	color=red,
	mark=diamond,
	]
	coordinates {
		(10000,6.778)(20000,13.393)(30000,20.482)(40000,30.896)(50000,40.196)(60000,53.579)(70000,58.858)(80000,69.323)(90000,80.632)(100000,93.133)(110000,102.264)(120000,113.286)(130000,128.353)(140000,142.335)(150000,159.782)(160000,173.954)(170000,207.902)(180000,187.756)(190000,206.874)(200000,220.576)
	};
	\addplot[
	color=green,
	mark=square,
	]
	coordinates {
		(10000,5.125)(20000,12.102)(30000,16.104)(40000,26.580)(50000,31.822)(60000,38.924)(70000,48.745)(80000,53.877)(90000,67.380)(100000,71.823)(110000,79.161)(120000,91.825)(130000,103.391)(140000,115.199)(150000,127.070)(160000,126.894)(170000,157.062)(180000,147.043)(190000,157.062)(200000,168.860)
	};
	\legend{
		$S_{SB}$,
		$S_{BB}$,
		$S_{L}$,
		$S_{TR}$,
		$S_{TB}$,
	}
	%%
	\end{axis}
	\end{tikzpicture}
\end{figure}
	
	
	\section{Wysokość drzewa}
%%WYKRES WYSOKOSCI DRZEW
	\begin{figure}[H]
		\subsubsection*{Wykres przedstawiający wysokości drzew od rozmiaru zestawu danych}
		\centering
		\begin{tikzpicture}
		\begin{axis}[
		width=0.9\textwidth,
		height = 0.5\textwidth,
		title={Wysokości drzew},
		xlabel={Liczba elementów},
		ylabel={Czas tworzenia w misekundach},
		%
		scaled x ticks = false,
		xtick distance = 10000,
		x tick label style={/pgf/number format/fixed},
		xticklabel style = {rotate= 90},
			x label style={at={(axis description cs:0.5,-0.15)},anchor=north},
		%%
		ytick distance = 2,
		scaled y ticks = false,
		y tick label style={/pgf/number format/fixed},
	y label style={at={(axis description cs:-0.05,0.85)},anchor=east},
		%%
		legend pos=north west,
		ymajorgrids=true,
		grid style=dashed,
		]
		%%
\addplot[
		color=red,
		mark=diamond,
		]
coordinates {		(10000,29.000)(20000,38.000)(30000,34.000)(40000,36.000)(50000,37.000)(60000,38.000)(70000,37.000)(80000,38.000)(90000,44.000)(100000,39.000)(110000,40.000)(120000,45.000)(130000,41.000)(140000,41.000)(150000,43.000)(160000,48.000)(170000,42.000)(180000,42.000)(190000,43.000)(200000,48.000)
		};
\addplot[
		color=green,
		mark=square,
		]
coordinates {
(10000,14.000)(20000,15.000)(30000,15.000)(40000,16.000)(50000,16.000)(60000,16.000)(70000,17.000)(80000,17.000)(90000,17.000)(100000,17.000)(110000,17.000)(120000,17.000)(130000,17.000)(140000,18.000)(150000,18.000)(160000,18.000)(170000,18.000)(180000,18.000)(190000,18.000)(200000,18.000)
};
		\legend{
			$H_{TR}$,
			$H_{TB}$,
		}
		%%
		\end{axis}
		\end{tikzpicture}
	\end{figure}
	
\end{spacing}
	\newpage
	\tableofcontents
\end{document}


