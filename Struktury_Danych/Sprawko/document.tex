\documentclass[polish,polish,a4paper]{article}
\usepackage[T1]{fontenc}
\usepackage[utf8]{inputenc}
\usepackage{pgfplots}
\usepackage{pslatex}
\usepackage{setspace}
\usepackage{caption}
\usepackage{amssymb}
\usepackage{amsmath}
\usepackage{anysize}
\usepackage{graphicx}
\usepackage{hyperref}
\usepackage{float}
\usepackage[polish]{babel}
\hypersetup{
	colorlinks=true,
	linkcolor=blue,
	filecolor=blue,      
	urlcolor=blue,
}

\marginsize{2.5cm}{2.5cm}{2cm}{2cm}


\begin{document}
	
		\begin{titlepage}
			\begin{flushright}
				{ Wtorki 16:50\\
					Grupa I3\\
					Kierunek Informatyka\\
					Wydział Informatyki\\
					Politechnika Poznańska}
			\end{flushright}
		\vspace*{\fill}
		\begin{center}
			{\Large Algorytmy i struktury danych \\[0.1cm]
				Sprawozdanie z zadania w zespołach nr. 2\\[0.1cm]
				prowadząca: dr hab. inż. Małgorzata Sterna, prof PP}\\
			{\Huge Wybrane złożone struktury danych\\ [0.4cm]}
			{\large autorzy:\\[0.1cm]}
			{\large Piotr Więtczak nr indeksu 132339\\[0.1cm] Tomasz Chudziak nr indeksu 136691}\\[0.5cm]
			\today
		\end{center}
		\vspace*{\fill}
	\end{titlepage}

\section{Opis implementacji i czasy tworzenia wybranych struktur danych}

	
Do implementacji wybranych struktur danych użyliśmy języka C++, a do pomiarów czasu klasy $ std::chrono::high\_resolution\_clock $  z biblioteki $<chrono>$.Wszystkie pomiary rozmiarów wykonano funkcją sizeof().

\subsection*{Proces generowania tablicy o unikalnych elementach}

Każdy element tablicy był losowany przy pomocy funkcji rand() z przedziału $ [0;100n]$, po wylosowaniu sprawdzano czy wylosowany element znajduje się już w tablicy pomocniczej przy pomocy wyszukiwania połówkowego, która zawierała te same elementy co tablica podstawowa, ale posortowane rosnąco. W przypadku kiedy wylosowany element znajdował się już w tablicy pomocniczej był losowany ponownie, jeżeli był to unikalny element zostawał dodany na koniec do tablicy podstawowej i pomocniczej, a następnie przesunięty na właściwe miejsce w tablicy pomocniczej. Cały proces był powtarzany aż tablica podstawowa osiągnęła pożądaną długość.

\subsection*{Tablica}

Tablica rozmiar pojedynczego elementu:
4 B

\subsection*{Lista jednokierunkowa}

Lista jednokierunkowa opiera się na dwóch rodzajach struktur. Pierwsza z nich to głowa, w niej znajdują się wskaźniki pokazujące na pierwszy i ostatni element listy, sama w sobie nie przechowuje żadnego elementu. Drugi rodzaj to element składa się on przechowywanej wartości oraz ze wskaźnika lokalizującego następny element. Każda lista posiada jedną głowę i pewną ilość elementów odpowiadającą ilości liczb do zapamiętania.
Rozmiar głowy w liście jednokierunkowej:
16 B.
Rozmiar elementu w liście jednokierunkowej:
16 B

\subsection*{Drzewo poszukiwań binarnych}
BST opiera się na jednej klasie. Podczas tworzenia tej struktury początkowo tworzy się tylko głowa to ona odpowiedzialna jest za wskazanie gdzie rozpoczyna się korzeń. By odróżnić ją od elementów przechowujących liczby do zapamiętania daliśmy jej wartość -1 (założyliśmy że ta konstrukcja będzie przechowywać tylko i wyłącznie liczby większe, równe 0). W klasie tej zaimplementowaliśmy konstruktor oraz trzy metody. Pierwsza przyłącza element do struktury. Podczas dodawania pierwszego elementu głowa zaczyna wskazywać na niego. Dodawanie kolejnych odbywa się tak samo jak dla zwykłego  BST z tą różnicą, że wpierw algorytm musi przejść przez głowę nie uwzględniając jej w dalszym szukaniu. Druga sprawdza wysokość drzewa, robi to w sposób rekurencyjny nie uwzględniając głowy. Trzecia odpowiedzialna jest za sprawdzanie czy element znajduje się w strukturze, polega ona na przeszukiwaniu tej struktury.
Rozmiar elementu w drzewie przeszukiwań binarnych:
32 B.


	\subsubsection*{Wykresy przedstawiające średni czas tworzenia struktury, w milisekundach, od rozmiaru zestawu danych}

%%WYKRES CZASU TWORZENIA
\begin{figure}[H]
	\centering
	\begin{tikzpicture}
	\begin{axis}[
	width=0.9\textwidth,
	height = 0.5\textwidth,
	title={Czasy tworzenia wybranych struktur danych},
	xlabel={Liczba elementów},
	ylabel={Czas tworzenia w misekundach},
	%
	scaled x ticks = false,
	xtick distance = 10000,
	x tick label style={/pgf/number format/fixed},
	xticklabel style = {rotate= 90},
		x label style={at={(axis description cs:0.5,-0.15)},anchor=north},
	%%
	ytick distance = 25,
	scaled y ticks = false,
	y tick label style={/pgf/number format/fixed},
	y label style={at={(axis description cs:-0.05,0.85)},anchor=east},
	%%
	legend pos=north west,
	ymajorgrids=true,
	grid style=dashed,
	]
	%%
\addplot[
	color=black,
	mark=*,
	]
	coordinates {
(10000,1.650)(20000,5.843)(30000,5.028)(40000,6.641)(50000,8.402)(60000,10.921)(70000,12.831)(80000,14.280)(90000,15.891)(100000,17.556)(110000,19.236)(120000,21.554)(130000,25.102)(140000,24.829)(150000,26.912)(160000,28.519)(170000,31.551)(180000,33.262)(190000,34.803)(200000,37.533)
	};
	%%
\addplot[
	color=blue,
	mark=triangle,
	]
	coordinates {
(10000,2.044)(20000,4.073)(30000,6.169)(40000,8.355)(50000,10.297)(60000,12.575)(70000,14.377)(80000,16.281)(90000,18.565)(100000,20.329)(110000,22.617)(120000,25.787)(130000,25.953)(140000,28.711)(150000,30.632)(160000,32.516)(170000,34.703)(180000,38.168)(190000,38.614)(200000,40.629)
	};
\addplot[
	color=red,
	mark=diamond,
	]
	coordinates {
(10000,9.945)(20000,22.923)(30000,32.196)(40000,46.658)(50000,57.789)(60000,76.041)(70000,94.813)(80000,102.981)(90000,125.854)(100000,130.962)(110000,154.307)(120000,170.075)(130000,184.363)(140000,202.444)(150000,215.974)(160000,245.860)(170000,280.478)(180000,264.447)(190000,285.930)(200000,303.575)
	};
\addplot[
	color=green,
	mark=square,
	]
	coordinates {
(10000,7.169)(20000,14.959)(30000,23.456)(40000,32.567)(50000,42.923)(60000,49.510)(70000,58.417)(80000,67.797)(90000,75.034)(100000,84.913)(110000,95.912)(120000,105.213)(130000,111.399)(140000,120.719)(150000,135.144)(160000,138.828)(170000,158.247)(180000,160.696)(190000,168.794)(200000,180.382)
	};
	\legend{
		$C_{B}$,
		$C_{L}$,
		$C_{TR}$,
		$C_{TB}$,
	}
	%%
	\end{axis}
	\end{tikzpicture}
\end{figure}

%%WYKRES CZASU TWORZENIA skala logarytmiczna
\begin{figure}[H]
	\centering
	\begin{tikzpicture}
	\begin{axis}[
	width=0.9\textwidth,
	height = 0.5\textwidth,
	title={Czasy tworzenia wybranych struktur danych, skala logarytmiczna},
	xlabel={Liczba elementów},
	ylabel={Czas tworzenia w misekundach},
	%
	scaled x ticks = false,
	xtick distance = 10000,
	x tick label style={/pgf/number format/fixed},
	xticklabel style = {rotate= 90},
	x label style={at={(axis description cs:0.5,-0.15)},anchor=north},
	%%
	ymode = log,
	scaled y ticks = false,
	y tick label style={/pgf/number format/fixed},
y label style={at={(axis description cs:-0.05,0.85)},anchor=east},
	%%
	legend pos=north west,
	ymajorgrids=true,
	grid style=dashed,
	]
	%%
	\addplot[
	color=black,
	mark=*,
	]
	coordinates {
		(10000,1.650)(20000,5.843)(30000,5.028)(40000,6.641)(50000,8.402)(60000,10.921)(70000,12.831)(80000,14.280)(90000,15.891)(100000,17.556)(110000,19.236)(120000,21.554)(130000,25.102)(140000,24.829)(150000,26.912)(160000,28.519)(170000,31.551)(180000,33.262)(190000,34.803)(200000,37.533)
	};
	%%
	\addplot[
	color=blue,
	mark=triangle,
	]
	coordinates {
		(10000,2.044)(20000,4.073)(30000,6.169)(40000,8.355)(50000,10.297)(60000,12.575)(70000,14.377)(80000,16.281)(90000,18.565)(100000,20.329)(110000,22.617)(120000,25.787)(130000,25.953)(140000,28.711)(150000,30.632)(160000,32.516)(170000,34.703)(180000,38.168)(190000,38.614)(200000,40.629)
	};
	\addplot[
	color=red,
	mark=diamond,
	]
	coordinates {
		(10000,9.945)(20000,22.923)(30000,32.196)(40000,46.658)(50000,57.789)(60000,76.041)(70000,94.813)(80000,102.981)(90000,125.854)(100000,130.962)(110000,154.307)(120000,170.075)(130000,184.363)(140000,202.444)(150000,215.974)(160000,245.860)(170000,280.478)(180000,264.447)(190000,285.930)(200000,303.575)
	};
	\addplot[
	color=green,
	mark=square,
	]
	coordinates {
		(10000,7.169)(20000,14.959)(30000,23.456)(40000,32.567)(50000,42.923)(60000,49.510)(70000,58.417)(80000,67.797)(90000,75.034)(100000,84.913)(110000,95.912)(120000,105.213)(130000,111.399)(140000,120.719)(150000,135.144)(160000,138.828)(170000,158.247)(180000,160.696)(190000,168.794)(200000,180.382)
	};
	\legend{
		$C_{B}$,
		$C_{L}$,
		$C_{TR}$,
		$C_{TB}$,
	}
	%%
	\end{axis}
	\end{tikzpicture}
\end{figure}

\begin{spacing}{1,5}
%%tabela czasów tworzenia
\begin{figure}[H]
		\subsubsection*{Tabela przedstawiająca średni czas tworzenia struktury w milisekundach}
		\centering
		\begin{equation*}
		\begin{array}{|r|r|r|r|r|}
		\hline
		\multicolumn{1}{|c|}{$Liczba ele.$}&\multicolumn{1}{c|}{C_{B}\enskip[ms]}&\multicolumn{1}{c|}{C_{L}\enskip[ms]}&\multicolumn{1}{c|}{C_{TR}\enskip[ms]}&\multicolumn{1}{c|}{C_{TB}\enskip[ms]}\\ \hline
		10000&1.650&2.044&9.945&7.169\\
		20000&5.843&4.073&22.923&14.959\\
		30000&5.028&6.169&32.196&23.456\\
		40000&6.641&8.355&46.658&32.567\\
		50000&8.402&10.297&57.789&42.923\\
		60000&10.921&12.575&76.041&49.510\\
		70000&12.831&14.377&94.813&58.417\\
		80000&14.280&16.281&102.981&67.797\\
		90000&15.891&18.565&125.854&75.034\\
		100000&17.556&20.329&130.962&84.913\\
		110000&19.236&22.617&154.307&95.912\\
		120000&21.554&25.787&170.075&105.213\\
		130000&25.102&25.953&184.363&111.399\\
		140000&24.829&28.711&202.444&120.719\\
		150000&26.912&30.632&215.974&135.144\\
		160000&28.519&32.516&245.861&138.828\\
		170000&31.551&34.703&280.478&158.247\\
		180000&33.262&38.168&264.447&160.696\\
		190000&34.803&38.614&285.932&168.794\\
		200000&37.533&40.629&303.575&180.382\\\hline		
		\end{array}
		\end{equation*}
\end{figure}
\end{spacing}

\section{Przeszukiwanie wybranych struktur danych}	

\subsection*{Wykresy przedstawiające średnie czasy wszystkich elementów struktury, w milisekundach, od rozmiaru zestawu danych}
%czasy przeszukiwania
	\begin{figure}[H]

		\centering
		\begin{tikzpicture}
		\begin{axis}[
		width=0.9\textwidth,
		height = 0.5\textwidth,
		title={Czasy przeszukiwania wybranych struktur danych},
		xlabel={Liczba elementów},
		ylabel={Czas przeszukiwania w misekundach},
		%
		scaled x ticks = false,
		xtick distance = 10000,
		x tick label style={/pgf/number format/fixed},
		xticklabel style = {rotate= 90},
			x label style={at={(axis description cs:0.5,-0.15)},anchor=north},
	%%	ymode = log,
		%%
		ytick distance = 25000,%%%
		scaled y ticks = false,
		y tick label style={/pgf/number format/fixed},
	y label style={at={(axis description cs:-0.05,0.85)},anchor=east},
		%%
		legend pos=north west,
		ymajorgrids=true,
		grid style=dashed,
		]
		%%
		\addplot[
		color=black,
		mark=*,
		]
		coordinates {
(10000,116.347)(20000,434.923)(30000,963.589)(40000,1663.180)(50000,2755.070)(60000,3933.720)(70000,5211.100)(80000,6811.880)(90000,8601.650)(100000,10571.600)(110000,12800.800)(120000,15197.600)(130000,17988.800)(140000,20721.400)(150000,23809.200)(160000,27484.000)(170000,30563.800)(180000,34389.200)(190000,38219.600)(200000,42337.700)
		};
		%%
		\addplot[
		color=gray,
		mark=*,
		]
		coordinates {
(10000,4.534)(20000,11.445)(30000,15.227)(40000,20.992)(50000,26.554)(60000,33.620)(70000,41.562)(80000,46.460)(90000,52.667)(100000,58.985)(110000,66.020)(120000,72.712)(130000,82.524)(140000,87.497)(150000,96.570)(160000,102.584)(170000,109.273)(180000,114.565)(190000,123.794)(200000,132.252)
		};
		\addplot[
		color=blue,
		mark=triangle,
		]
		coordinates {
(10000,354.608)(20000,1435.100)(30000,3150.970)(40000,5964.640)(50000,17094.000)(60000,16052.900)(70000,21362.700)(80000,34108.400)(90000,45225.500)(100000,64175.800)(110000,79715.500)(120000,102809.000)(130000,131624.000)(140000,159801.000)(150000,183256.000)(160000,222896.000)(170000,244782.000)(180000,292448.000)(190000,318782.000)(200000,371754.000)
		};
\addplot[
		color=red,
		mark=diamond,
		]
		coordinates {
(10000,6.778)(20000,13.393)(30000,20.482)(40000,30.896)(50000,40.196)(60000,53.579)(70000,58.858)(80000,69.323)(90000,80.632)(100000,93.133)(110000,102.264)(120000,113.286)(130000,128.353)(140000,142.335)(150000,159.782)(160000,173.954)(170000,207.902)(180000,187.756)(190000,206.874)(200000,220.576)
		};
\addplot[
	color=green,
	mark=square,
	]
	coordinates {
(10000,5.125)(20000,12.102)(30000,16.104)(40000,26.580)(50000,31.822)(60000,38.924)(70000,48.745)(80000,53.877)(90000,67.380)(100000,71.823)(110000,79.161)(120000,91.825)(130000,103.391)(140000,115.199)(150000,127.070)(160000,126.894)(170000,157.062)(180000,147.043)(190000,157.062)(200000,168.860)
	};
		\legend{
			$S_{SB}$,
			$S_{BB}$,
			$S_{L}$,
			$S_{TR}$,
			$S_{TB}$,
		}
		%%
		\end{axis}
		\end{tikzpicture}
	\end{figure}
%czasy przeszukiwania skala logarytmiczna
	\begin{figure}[H]
	
	\centering
	\begin{tikzpicture}
	\begin{axis}[
	width=0.9\textwidth,
	height = 0.5\textwidth,
	title={Czasy przeszukiwania wybranych struktur danych, skala logarytmiczna},
	xlabel={Liczba elementów},
	ylabel={Czas przeszukiwania w misekundach},
	%
	scaled x ticks = false,
	xtick distance = 10000,
	x tick label style={/pgf/number format/fixed},
	xticklabel style = {rotate= 90},
		x label style={at={(axis description cs:0.5,-0.15)},anchor=north},
	ymode = log,
	%%
	scaled y ticks = false,
	y tick label style={/pgf/number format/fixed},
y label style={at={(axis description cs:-0.05,0.85)},anchor=east},
	%%
	legend pos=north west,
	ymajorgrids=true,
	grid style=dashed,
	]
	%%
	\addplot[
	color=black,
	mark=*,
	]
	coordinates {
		(10000,116.347)(20000,434.923)(30000,963.589)(40000,1663.180)(50000,2755.070)(60000,3933.720)(70000,5211.100)(80000,6811.880)(90000,8601.650)(100000,10571.600)(110000,12800.800)(120000,15197.600)(130000,17988.800)(140000,20721.400)(150000,23809.200)(160000,27484.000)(170000,30563.800)(180000,34389.200)(190000,38219.600)(200000,42337.700)
	};
	%%
	\addplot[
	color=gray,
	mark=*,
	]
	coordinates {
		(10000,4.534)(20000,11.445)(30000,15.227)(40000,20.992)(50000,26.554)(60000,33.620)(70000,41.562)(80000,46.460)(90000,52.667)(100000,58.985)(110000,66.020)(120000,72.712)(130000,82.524)(140000,87.497)(150000,96.570)(160000,102.584)(170000,109.273)(180000,114.565)(190000,123.794)(200000,132.252)
	};
	\addplot[
	color=blue,
	mark=triangle,
	]
	coordinates {
		(10000,354.608)(20000,1435.100)(30000,3150.970)(40000,5964.640)(50000,17094.000)(60000,16052.900)(70000,21362.700)(80000,34108.400)(90000,45225.500)(100000,64175.800)(110000,79715.500)(120000,102809.000)(130000,131624.000)(140000,159801.000)(150000,183256.000)(160000,222896.000)(170000,244782.000)(180000,292448.000)(190000,318782.000)(200000,371754.000)
	};
	\addplot[
	color=red,
	mark=diamond,
	]
	coordinates {
		(10000,6.778)(20000,13.393)(30000,20.482)(40000,30.896)(50000,40.196)(60000,53.579)(70000,58.858)(80000,69.323)(90000,80.632)(100000,93.133)(110000,102.264)(120000,113.286)(130000,128.353)(140000,142.335)(150000,159.782)(160000,173.954)(170000,207.902)(180000,187.756)(190000,206.874)(200000,220.576)
	};
	\addplot[
	color=green,
	mark=square,
	]
	coordinates {
		(10000,5.125)(20000,12.102)(30000,16.104)(40000,26.580)(50000,31.822)(60000,38.924)(70000,48.745)(80000,53.877)(90000,67.380)(100000,71.823)(110000,79.161)(120000,91.825)(130000,103.391)(140000,115.199)(150000,127.070)(160000,126.894)(170000,157.062)(180000,147.043)(190000,157.062)(200000,168.860)
	};
	\legend{
		$S_{SB}$,
		$S_{BB}$,
		$S_{L}$,
		$S_{TR}$,
		$S_{TB}$,
	}
	%%
	\end{axis}
	\end{tikzpicture}
\end{figure}

\begin{spacing}{1,5}
%%tabela czasów przeszukiwania	
\begin{figure}[H]
	\subsubsection*{Tabela przedstawiająca czas wyszukiwania wszystkich elementów struktury w milisekundach}
	\centering
	\begin{equation*}
	\begin{array}{|r|r|r|r|r|r|}
	\hline
	\multicolumn{1}{|c|}{$Liczba ele.$}&\multicolumn{1}{c|}{S_{SB}\enskip[ms]}&\multicolumn{1}{c|}{S_{BB}\enskip[ms]}&\multicolumn{1}{c|}{S_{L}\enskip[ms]}&\multicolumn{1}{c|}{S_{TR}\enskip[ms]}&\multicolumn{1}{c|}{S_{TB}\enskip[ms]}\\ \hline
10000&116.347&4.534&354.608&6.778&5.125\\
20000&434.923&11.445&1435.100&13.393&12.102\\
30000&963.589&15.227&3150.970&20.482&16.104\\
40000&1663.180&20.992&5964.640&30.896&26.580\\
50000&2755.070&26.554&17094.000&40.196&31.822\\
60000&3933.720&33.620&16052.900&53.579&38.924\\
70000&5211.100&41.562&21362.700&58.858&48.745\\
80000&6811.880&46.460&34108.400&69.323&53.877\\
90000&8601.650&52.667&45225.500&80.632&67.380\\
100000&10571.600&58.985&64175.800&93.133&71.823\\
110000&12800.800&66.020&79715.500&102.264&79.161\\
120000&15197.600&72.712&102809.000&113.286&91.825\\
130000&17988.800&82.524&131624.000&128.353&103.391\\
140000&20721.400&87.497&159801.000&142.335&115.199\\
150000&23809.200&96.570&183256.000&159.782&127.070\\
160000&27484.000&102.584&222896.000&173.954&126.894\\
170000&30563.800&109.273&244782.000&207.902&157.062\\
180000&34389.200&114.565&292448.000&187.756&147.043\\
190000&38219.600&123.794&318782.000&206.874&157.062\\
200000&42337.700&132.252&371754.000&220.576&168.860\\\hline
	\end{array}
	\end{equation*}
\end{figure}
\end{spacing}
	
Wyszukiwanie elementów w tablicy metodą wyczerpującą dla wszystkich elementów nie różni się złożonością, która wynosi $O(n^2)$,  od wyszukiwania w liście jednokierunkowej. Działają one na podobnej zasadzie, dla każdego wyszukania, sprawdzane są wszystkie elementy od początkowego, aż do szukanego. Mimo to przeszukiwanie listy okazało się wolniejsze, a wraz ze wzrostem liczby elementów różnica pogłębia się. Dla 10000 elementów wyniki dla tablicy są ponad dwa razy gorsze, dla 100000 sześć i pół razy, a dla 200000 prawie 9 razy. Powodem tego może być to że lista odwołuje się do elementów po adresie w pamięci, a nie jak w przypadku tablicy po numerze indeksu.
Przeszukiwanie  tablicy za pomocą wyszukiwania binarnego było znacznie szybsze od wyszukiwania w liście, na początku radziło sobie około 80 razy lepiej, a pod koniec aż blisko 2800 razy.

Czasy otrzymane dla przeszukiwania binarnego tablicy B i drzewa TB, osiągały podobne wartości, jednak we wszystkich przypadkach wyszukiwanie w tablicy poradziło sobie lepiej, zapewne z tego samego powodu co w przypadku listy poruszanie się po strukturze za pomocą adresów poszczególnych elementów jest mniej wydajne od używania ich indeksów. W obu przypadkach przebieg wyszukiwania wszystkich elementów wygląda podobnie, za każdym razem sprawdzane jest czy poszukiwany element jest równy aktualnie wybranemu, w przeciwny razie dzięki uporządkowaniu struktur wyszukiwanie jest kontynuowane tyko dla mniejszych/większych od wybranego, co pozwala na zmniejszenie mocy zbioru w którym poszukujemy o połowę. Złożoność takiego wyszukiwania dla wszystkich elementów wynosi $O(n\log_{n})$.

	\section{Wysokość drzewa poszukiwań binarnych}
%%WYKRES WYSOKOSCI DRZEW
	\begin{figure}[H]
		\subsubsection*{Wykres przedstawiający wysokości drzew od rozmiaru zestawu danych}
		\centering
		\begin{tikzpicture}
		\begin{axis}[
		width=0.9\textwidth,
		height = 0.5\textwidth,
		title={Wysokości drzew},
		xlabel={Liczba elementów},
		ylabel={Czas tworzenia w misekundach},
		%
		scaled x ticks = false,
		xtick distance = 10000,
		x tick label style={/pgf/number format/fixed},
		xticklabel style = {rotate= 90},
			x label style={at={(axis description cs:0.5,-0.15)},anchor=north},
		%%
		ytick distance = 2,
		scaled y ticks = false,
		y tick label style={/pgf/number format/fixed},
	y label style={at={(axis description cs:-0.05,0.85)},anchor=east},
		%%
		legend pos=north west,
		ymajorgrids=true,
		grid style=dashed,
		]
		%%
\addplot[
		color=red,
		mark=diamond,
		]
coordinates {		(10000,29.000)(20000,38.000)(30000,34.000)(40000,36.000)(50000,37.000)(60000,38.000)(70000,37.000)(80000,38.000)(90000,44.000)(100000,39.000)(110000,40.000)(120000,45.000)(130000,41.000)(140000,41.000)(150000,43.000)(160000,48.000)(170000,42.000)(180000,42.000)(190000,43.000)(200000,48.000)
		};
\addplot[
		color=green,
		mark=square,
		]
coordinates {
(10000,14.000)(20000,15.000)(30000,15.000)(40000,16.000)(50000,16.000)(60000,16.000)(70000,17.000)(80000,17.000)(90000,17.000)(100000,17.000)(110000,17.000)(120000,17.000)(130000,17.000)(140000,18.000)(150000,18.000)(160000,18.000)(170000,18.000)(180000,18.000)(190000,18.000)(200000,18.000)
};
		\legend{
			$H_{TR}$,
			$H_{TB}$,
		}
		%%
		\end{axis}
		\end{tikzpicture}
	\end{figure}
\subsection*{Tabela przedstawiająca wysokości drzew od liczby elementów}
\begin{spacing}{1,5}
%tabela wysokosci
\begin{figure}[H]

	\centering
	\begin{equation*}
	\begin{array}{|r|r|r|}
	\hline
	\multicolumn{1}{|c|}{$Liczba ele.$}&\multicolumn{1}{c|}{H_{TR}}&\multicolumn{1}{c|}{H_{TB}}\\ \hline
10000&29&14\\
20000&38&15\\
30000&34&15\\
40000&36&16\\
50000&37&16\\
60000&38&16\\
70000&37&17\\
80000&38&17\\
90000&44&17\\
100000&39&17\\
110000&40&17\\
120000&45&17\\
130000&41&17\\
140000&41&18\\
150000&43&18\\
160000&48&18\\
170000&42&18\\
180000&42&18\\
190000&43&18\\
200000&48&18\\\hline
	\end{array}
	\end{equation*}
\end{figure}
\end{spacing}

Wysokość badanych drzew TR okazuje się być ponad dwukrotnie większa od drzew TB. Do drzewa TR wprowadzane są dane o losowych wartościach, przez co wysokość jest większa od wartości optymistycznej $O(\log{n})$. Natomiast do drzewa TB wprowadzane dane pochodzą z podziału połówkowego posortowanej tablicy, stworzone tak drzewo będzie dokładnie wyważone, a jego wysokość optymalna.
Gdyby dane wprowadzano do BST według tablicy B, czyli posortowane rosnącą, drzewo przyjęło by postać jednej gałęzi o długości odpowiadającej liczbie wprowadzonych elementów, jego wysokość osiągnęła by wartość pesymistyczną $ O(n) $.

Wysokość drzewa przeszukiwań binarnych ma bardzo duże znaczenie. To właśnie od niej zależy maksymalna ścieżka jaką musi przejść algorytm w poszukiwaniu elementów czy  podczas próby dodania nowego. 
Najkorzystniejsze jest drzewo wyważone. Ma to związek z wysokością struktury, średni i optymistyczny czas przeszukiwania będzie wynosił $O(\log_{n})$. Najmniej korzystne jest drzewo zdegenerowane do listy, oznacza to, że wszystkie elementy maja wpływ na wysokość drzewa, to zjawisko można osiągnąć wprowadzając do drzewa dane w posortowanej kolejności. Dla tego przypadku czas wyszukiwania zwiększa się do $ O(n) $. Oznacza to, że struktura będzie działać najwydajniej gdy jest wyważona. Zmniejsza to średni dostęp do wyszukiwanego elementu.

\section{Wady i zalety wybranych struktur danych}
\subsection{Tablica:}
Zalety:
\begin{itemize}
	\item krótki czas przeszukiwania binarnego
	\item dostęp do elementu po numerze indeksu
	\item łatwa implementacja (można skorzystać ze struktury standardowo zaimplementowanej w większości języków programowania)
	\item krótki czas tworzenia
	\item niskie wymagania pamięciowe
\end{itemize}
Wady:
\begin{itemize}
	\item długi czas przeszukiwania wycieńczającego
	\item mała elastyczność (utrudniona możliwość dodania i usuwania elementów)
\end{itemize}


\subsection{Lista jednokierunkowa:}
Zalety:
\begin{itemize}
	\item krótki czas tworzenia
	\item stosunkowo łatwa w implementacji 
	\item duża elastyczność (łatwe dodawanie i usuwanie elementów)
\end{itemize}
Wady:
\begin{itemize}
	\item długi czas przeszukiwania
	\item brak dostępu do elementu po numerze indeksu
	\item wyższe wymagania pamięciowe od tablicy
\end{itemize}

\subsection{Drzewo poszukiwań binarnych}
Zalety:
\begin{itemize}
	\item krótki czas przeszukiwania w przypadku optymistycznym
	\item średnia elastyczność (możliwość dodawania i usuwania elementów)
\end{itemize}
Wady:
\begin{itemize}
	\item brak dostępu do elementu po numerze indeksu
	\item największe wymagania pamięciowe z badanych struktur
	\item najtrudniejsza w implementacji z badanych struktur
	\item w przypadku pesymistycznym złożoność przeszukiwania wynosi $O(n)$
	\item średnia elastyczność (dodanie i usuwanie elementów jest trudniejsze do implementacji niż w liście jednokierunkowej i może mieć wysoką złożoność dla elementów znajdujący się daleko od korzenia)
	\item wrażliwa na dane wejściowe (w przypadku pesymistycznym, kiedy dane wejściowe są posortowane, drzewo przyjmie formę zdegradowanej lisy, złożoność tworzenia osiąga $O(n^{2})$)
\end{itemize}

	\newpage
	\tableofcontents
\end{document}


